% \mysection{6}{RESULTADOS}
\section{RESULTADOS}

Se propuso una metodología para diseñar microservicios en base a la literatura revisada.
Para la propuesta de diseño de microservicios, este trabajo utiliza como base el Diseño Guiado por el Dominio
agregando entradas, técnicas y salidas específicas para aclarar el proceso de manera tal que
sea aplicable a diferentes contextos sin abrumar al arquitecto/equipo de desarrollo.

La arquitectura producto de seguir el proceso se acomoda a los objetivos planteados.
Los microservicios producto de la metodología propuesta son escalables horizontalmente porque
los sub-dominios en los que se basan están separados en los aspectos lógicos y de datos.
El producto final es fiable porque definiendo contextos delimitados se hace el análisis del flujo
de datos con el fin de obtener las relaciones del flujo de datos de cada sub-dominio.
La arquitectura resultante cumple con el despliegue independiente de cada componente debido
a la alta cohesión de cada servicio y el bajo acoplamiento entre servicios debido a que el
modelo del dominio y los sub-dominios identificados son analizados exhaustivamente con una
perspectiva holística con las diferentes técnicas a lo largo del proceso.


% Para la arquitectura resultante de la metodología propuesta tiene un balance entre complejidad
% y facilidad de implementación
