% \mysection{1}{INTRODUCCIÓN}
\mysection{INTRODUCCIÓN}

En la actualidad los requerimientos de los sistemas han cambiado.
Cada vez más las empresas tienen la necesidad de procesar más y más data que sistemas de
una arquitectura monolítica no pueden atender de manera eficiente debido a las restricciones
de escalabilidad vertical y del tamaño de los equipos que mantienen estas aplicaciones.

Para afrontar estos nuevos retos en la ingeniería de software, nacieron diferentes herramientas
y disciplinas para poder manejar sistemas distribuidos capaces de escalar horizontalmente con costos
de infraestructura más flexibles.

En este contexto, la arquitectura de microservicios se consolida como la respuesta a necesidades
de escalabilidad, fiabilidad y despliegue independiente que se exige de sistemas modernos.
Además de los beneficios antes mencionados, los microservicios hacen que la organización de equipos
sea más efectiva, pudiendo organizar múltiples equipos ágiles multidisciplinarios para atender
todas las necesidades que el dominio del negocio pueda requerir.

La motivación detrás de esta monografía es por la falta de una guía completa de cómo
organizar una aplicación en microservicios.
Debido a esta fragmentación de la información se hace la revisión sistemática de
la literatura para poder tener un panorama claro del cómo plantear una arquitectura de microservicios.

\newpage
