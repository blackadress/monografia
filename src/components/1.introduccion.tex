% \mysection{1}{INTRODUCCIÓN}
\mysection{INTRODUCCIÓN}

La arquitectura de microservicios es una forma de organizar los servicios web que responde a las
necesidades modernos de sistemas.
Este tipo de arquitectura es capaz de atender solicitudes HTTP de grandes volúmenes además de ser
fácilmente desplegadas bajo demanda.
Junto con las ventajas tecnológicas que esta arquitectura ofrece, también tiene la ventaja de poder
escalar equipos ágiles de desarrollo de manera prácticamente ilimitada.

En esta monografía no se tocarán las partes del diseño táctico del Diseño Guiado por el Dominio debido a
que el análisis de este nivel siempre es diferente dependiendo del Dominio que se analiza.

El trabajo cuenta con tres bloques, el Capítulo I donde se definen los objetivos de esta monografía.
El Capítulo II de marco teórico.
El Capítulo III dónde se define la metodología propuesta para hacer el diseño de microservicios.
El Capítulo IV dónde se mencionan los resultados de la monografía.
Y finalmente el Capítulo V Donde se muestran las conclusiones del trabajo.

En el marco teórico se explora textos de diseño de microservicios, teniendo más peso
los libros discutiendo el rediseño de arquitecturas monolíticas a arquitecturas de microservicios
debido a que casos como este son los más comunes en la industria.
En segundo orden, se hizo la revisión de literatura con respecto al Diseño Guiado por el Dominio
por ser la manera que la mayoría de arquitectos utilizan para diseñar microservicios.

Este trabajo tiene por objetivo el
diseñar una arquitectura de microservicios para servicios web.
Siendo los objetivos específicos:
1) Diseñar una arquitectura de microservicios para poder escalar los servicios web, 2022.
2) Diseñar una arquitectura de microservicios para mejorar la fiabilidad de los diferentes servicios web, 2022.
3) Diseñar una arquitectura de microservicios para el despliegue independiente de los diferentes servicios web, 2022.


\newpage
