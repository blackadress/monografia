% \mysection{7}{CONCLUSIONES}
\section{CONCLUSIONES}

En base a lo propuesto en el Capítulo de III, podemos afirmar que se logró
diseñar una arquitectura de microservicios escalable, fiable y de despliegue independiente.

De acuerdo a lo mencionado en el numeral 3.2.1 y 3.2.2, del Capítulo III, se logró que los
microservicios puedan ser desplegados de manera independiente.
En el proceso del análisis del dominio y la definición de contextos delimitados
el se modela el dominio de negocio de forma tal que cada componente de la arquitectura resultante
posee alta cohesión y el acoplamiento intercomponente es bajo.

De acuerdo al numeral 3.2.3, del Capítulo III, se logró que los
servicios web sean escalables.
En el proceso de identificación de microservicios, los servicios resultantes
pueden ser escalados horizontalmente debido al bajo acoplamiento que presentan.

De acuerdo a lo mencionado en el numeral 3.3, del Capítulo III, se logró una arquitectura
fiable.
En el proceso de diseñar un API gateway se implementa la lógica encargada de asegurar que
las solicitudes que lleguen tengan respuesta utilizando implementaciones de cola de solicitudes
y repetición de solicitudes.
