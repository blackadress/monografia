% \mysection{5}{METODOLOGÍA}
\section{METODOLOGÍA}

Para poder diseñar una arquitectura de microservicios efectiva, correcta y que responda a las
necesidades del negocio se propone una metodología iterativa que se compone de cinco etapas:


\subsection{Identificar el Dominio del Negocio}

Puede sonar como un axioma sin sentido, después de todo ¿qué negocio no sabe cuál es su dominio?
Sin embargo muchos arquitectos expertos siempre aconsejan empezar por aquí debido a que si bien los
expertos del dominio conocen la parte de negocio en la cual se desenvuelven, es altamente probable
que no conozcan los detalles del cómo funciona el negocio fuera de sus actividades cotidianas.

En concreto, un empleado del departamento de ventas puede conocer muy bien sus responsabilidades
y funciones dentro del flujo de ventas.
Además conoce relativamente bien las responsabilidades y funciones de los departamentos con los que más
interacción tiene (para este ejemplo almacén).
Sin embargo no conoce de los detalles de las funciones que estos otros departamentos cumplen, específicamente
no conoce de la cantidad exacta de productos que existen físicamente en almacén, no conoce los horarios
de reaprovisionamiento, no sabe que desde la semana pasada almacén tiene un nuevo vehículo que hará que
el trabajo sea más rápido.

El escenario mencionado se repite departamento a departamento, unidad de trabajo a unidad de trabajo
a lo largo y ancho de toda la organización.
Es debido a esto que es probable que la organización no tenga en claro específicamente qué acciones
son las que entregan valor y por lo tanto identificar el dominio del negocio debe ser el punto de inicio
para el equipo de desarrollo.

\subsubsection*{Entradas}
\begin{enumerate}[a.]
	\item Documentos internos de la empresa.
\end{enumerate}

\subsubsection*{Técnicas y Herramientas}
\begin{enumerate}[a.]
	\item Reuniones multi-departamentales.
	\item Revisión de documentos internos.
	\item Seguimiento de flujo de producto.
\end{enumerate}

\subsubsection*{Salidas}
En esta etapa no existen salidas concretas debido a que lo único que se quiere es que el equipo
de desarrollo entienda en un nivel alto el dominio de la empresa u organización.

\subsection{Diseño Guiado por el Dominio}

En esta etapa utilizaremos las pautas que nos da el diseño basado en dominios.
Se utiliza este marco porque ya fue probado múltiples veces en el mundo real para ir desde
el nivel más alto de la aplicación hasta definir los detalles de organización de los microservicios.

Debido a que se propone una metodología iterativa para refinar los microservicios, el marco de trabajo
de diseño basado en dominios tiene la ventaja de que también está pensado como un proceso iterativo
por lo cual es congruente a la metodología propuesta.

\subsubsection{Análisis del Dominio}

El equipo de desarrollo ya tiene una idea general de qué es el dominio del negocio, pero no tiene
nada formal con lo que empezar a implementar la solución.
Entonces lo que se debe hacer es crear un modelo del dominio.
Este modelo permite ver el dominio del negocio de manera holística para poder identificar
los sub-dominios núcleo, los subdominios de soporte y los subdominios genéricos.

Para identificar correctamente el modelo de dominio de la organización es necesaria la participación
de todos los departamentos.
La gerencia debe apoyar para garantizar la participación de toda la organización en el modelado del dominio.

Se requiere una participación total debido a que si bien cada departamento, oficina o equipo son expertos
en sus dominios aislados y sea tentador el modelar el dominio basándose en las divisiones que la organización ya
tiene, el modelo corre peligro de ser contradictorio y el diccionario de términos estar incompleto.

Si se comete el error de intentar modelar el dominio departamento por departamento, no se encontrarán
conflictos de flujo interdepartamentales, también es altamente probable que en el modelo de dominio
se separen elementos que realmente tienen alta cohesión generando problemas de dependencia en el futuro.
Además de los problemas que genera al modelo, el diccionario de términos que sirve para que el equipo
de desarrollo y los expertos del dominio puedan comunicarse efectivamente, quede incompleto o tenga
términos incorrectos o tenga términos contradictorios entre departamentos o en el peor de los casos
todo lo anterior mencionado.

\subsubsection*{Entradas}
\begin{enumerate}[a.]
	\item Flujo de valor informal de la organización.
\end{enumerate}

\subsubsection*{Técnicas y herramientas}
\begin{enumerate}[a.]
  \item Event storming
	\item Cuadros de dominios.
\end{enumerate}

% \subsubsection*{Ejemplo de la técnica event storming}
\vspace{1em}

  \begin{figure}[H]
    % \caption{\\\hspace{\textwidth} Ejemplo de un cuadro de dominio}
    \caption{Ejemplo de un cuadro producto de una reunión de event storming}
    \begin{center}
    \includegraphics[width=0.90\textwidth]{src/assets/metodologia/event_storming}
    \label{fig:event_storming}
    \end{center}
    Nota: Gráfica de ejemplo de un cuadro producto de una reunión de event storming.
    En este ejemplo se está haciendo la simulación de los eventos de una empresa de e-commerce.
    Elaboración propia.
  \end{figure}

% \subsubsection*{Ejemplo de la técnica cuadro de dominio}
\vspace{1em}

  \begin{figure}[H]
    % \caption{\\\hspace{\textwidth} Ejemplo de un cuadro de dominio}
    \caption{Ejemplo de un cuadro de dominio}
    \centering
    \includegraphics[width=0.90\textwidth]{src/assets/metodologia/cuadro_dominio}
    \label{fig:cuadro_dominio}
    \\
    Nota: Gráfica de ejemplo de un cuadro de dominio. Elaboración propia.
  \end{figure}


\subsubsection*{Salidas}
\begin{enumerate}[a.]
    \item Identificación de sub-dominios núcleo.
    \item Identificación de sub-dominios de soporte.
    \item Identificación de sub-dominios genéricos.
    \item Modelo de dominio.
    \item Diccionario de términos.
\end{enumerate}


\subsubsection{Definir Contextos Delimitados}
Luego de identificar los tres tipos de sub-dominios, ya se tiene una visión holística
del sistema que se va a desarrollar pero el modelo de dominio resultante todavía no está
lo suficientemente refinado como para iniciar el desarrollo del sistema.
El dominio todavía tiene relaciones de dependencia, eventos todavía están fuertemente ligados
entre sí, no existe una clara concepción de cómo separar este modelo en microservicios.

De la etapa anterior podemos descartar los sub-dominios genéricos porque pueden ser tercerizados
o en su defecto la organización puede adquirir una solución ya hecha para manejar estos sub-dominios.

Es debido a eso que tomando como base el modelo de dominio se deben definir los contextos delimitados
del modelo.
Para hacer esto hay que apoyarse en las reuniones con los expertos del dominio para identificar
límites entre procesos de la organización.
Una forma para encontrar estos límites lógicos es tomar en cuenta la semántica de las palabras usada por los 
expertos de dominio.

\subsubsection*{Entradas}
\begin{enumerate}[a.]
	\item Sub-dominios núcleo.
	\item Sub-dominios de soporte.
	\item Modelo de dominio.
\end{enumerate}

\subsubsection*{Técnicas y herramientas}
\begin{enumerate}[a.]
	\item Análisis semántico.
  \item Lienzo de contextos delimitados.
\end{enumerate}

\subsubsection*{Ejemplo de la técnica de análisis semántico}

En una entidad financiera el equipo comercial se puede referir a los créditos 
como 'operaciones' mientras que el equipo de cobranzas se refiere a la recaudación de la deuda como 'operaciones'.
Aquí en la misma organización se identifica que la palabra 'operaciones' tiene una semántica diferente entre
estas dos oficinas por lo que hace que se puedan identificar los límites en el modelo de dominio y a su vez 
encontrar los contextos delimitados.


% \subsubsection*{Ejemplo de la técnica de lienzo de contextos delimitados}
\vspace{1em}

  \begin{figure}[H]
    % \caption{\\\hspace{\textwidth} Ejemplo de un cuadro de dominio}
    \caption{Ejemplo de un lienzo de contextos delimitados (para un elemento)}
    \begin{center}
    \includegraphics[width=0.90\textwidth]{src/assets/metodologia/bounded_context_canvas}
    \label{fig:bounded_context_canvas}
    \end{center}
    \textit{Nota:} Adaptado de \textit{Bounded Context Canvas}, de \cite{canvas}, Github (shorturl.at/iPVW7). CC-BY-SA-4.0.
  \end{figure}

\subsubsection*{Salidas}
\begin{enumerate}[a.]
    \item Contextos delimitados.
    \item Relaciones de flujos de datos.
\end{enumerate}


\subsubsection{Identificar los Microservicios}

Ahora que ya se identificaron los contextos delimitados, ya se pueden identificar los microservicios
juntos formarán nuestra aplicación.
La palabra clave en esta etapa es "identificar", para tener una arquitectura de microservicios que soporte
el despliegue independiente y respete principios de cohesión no se puede forzar definiciones sobre el
dominio de negocio que ya existe, sino que se tienen que identificar y tienen que responder a la realidad.

Utilizando los contextos delimitados obtenidos con anterioridad se inicia el proceso de identificación
de microservicios.
En concreto, un microservicio no puede sobrepasar los límites impuestos por el contexto delimitado.
Cada microservicio debe tener una única responsabilidad siguiendo el principio de responsabilidad única,
que indica que un componente debe ser responsable de un solo usuario o stakeholder.

Finalmente para definir nuestros microservicios se debe tener en cuenta los requerimientos no funcionales.
Como factor más importante se debe tomar en cuenta el tamaño del equipo para cada microservicio, los tipos
de datos y los requisitos de escalabilidad y fiabilidad de cada microservicio.

\subsubsection*{Entradas}
\begin{enumerate}[a.]
  \item Contextos delimitados.
  \item Relaciones de flujos de datos.
\end{enumerate}

\subsubsection*{Técnicas y herramientas}
\begin{enumerate}[a.]
  \item Análisis de contextos delimitados.
  \item Análisis de requerimientos no funcionales.
\end{enumerate}

% \subsubsection*{Ejemplo de la técnica de análisis de contextos delimitados}
\vspace{1em}
\begin{figure}[H]
  \caption{Ejemplo de análisis de contextos delimitados}
  \begin{center}
    \includegraphics[width=0.90\textwidth]{src/assets/metodologia/bounded_context}
    \label{fig:bounded_context_example}
  \end{center}
  \textit{Nota:} Adaptado de \textit{BoundedContext}, de \cite{bounded} martinfowler.com (shorturl.at/aeHK9). CC-BY-SA-4.0.
\end{figure}

\subsubsection*{Salidas}
\begin{enumerate}
  \item Microservicios identificados.
\end{enumerate}


\subsection{Diseñar una API Gateway Como Punto de Entrada}
Una vez se tengan los microservicios identificados se presenta otro problema que tiene que ver con
la integración de los microservicios con sus clientes.
En el mundo actual, una organización puede tener múltiples aplicaciones clientes que dependen
de sus servicios backend. 
El comunicar estos clientes con el microservicio capaz de responder a la solicitud enviada
presenta la dificultad de mantener un registro de las direcciones hacia donde mandar las
solicitudes HTTP.

Además de el problema de coordinar todos los puntos de entrada entre múltiples equipos,
la granularidad de las APIs expuestas por los microservicios suelen ser diferentes
a las necesidades específicas de cada cliente.
Los microservicios suelen proveer interfaces muy granuladas por lo que es muy común que
requieran interactuar con múltiples servicios para obtener la data requerida.

Debido la complejidad de llamar múltiples servicios y procesar esta data en el terreno de usuario,
se presenta la solución de un API gateway que provee de un único punto de entrada a la aplicación. 
Con la implementación de un API gateway, los clientes solo tienen que tener identificados un solo
dominio a donde mandar las solicitudes HTTP, además de que los endpoints expuestos por la API gateway
pueden consolidar grupos de servicios dando respuestas específicas a solicitudes de clientes.

En resumen la implementación de una API gateway simplifica la comunicación con los posibles clientes
de la aplicación además de poder asegurar la fiabilidad de los servicios implementando
una cola de solicitudes así como también repetir solicitudes en caso alguna de estas fallen.
La desventaja de esta solución está en la complejidad de la capa de comunicación.
Debido a que es posible consolidar varios servicios en un solo endpoint, es muy probable
que la aplicación vaya a tener duplicidad accidental entre endpoints.
Para mitigar estos problemas se sugiere en primera instancia utilizar herramientas de
análisis estático o dividir la API gateway por cliente.


\subsubsection*{Entradas}
\begin{enumerate}[a.]
  \item Microservicios identificados.
\end{enumerate}

\subsubsection*{Técnicas y herramientas}
\begin{enumerate}[a.]
  \item Análisis documental de los requerimientos de data de aplicaciones cliente.
  \item Reuniones con los equipos encargados de las aplicaciones cliente.
\end{enumerate}

\subsubsection*{Salidas}
\begin{enumerate}
  \item Diseño de alto nivel del API gateway
\end{enumerate}

% \subsubsection*{Ejemplo del diseño de una API gateway}
\vspace{1em}
\begin{figure}[H]
  \caption{Ejemplo de diseño de una API gateway}
  \begin{center}
    \includegraphics[width=0.90\textwidth]{src/assets/metodologia/apigateway}
    \label{fig:api_gateway}
  \end{center}
  \textit{Nota:} Adaptado de \textit{Microservices patterns> with examples in Java}, de \cite{richardson2018microservices}
  (p. 226). CC-BY-SA-4.0.
\end{figure}

% \subsubsection*{Ejemplo del diseño de API gateways divididas por cliente}
\vspace{1em}
\begin{figure}[H]
  \caption{Ejemplo de diseño de una API gateways divididas por clientes}
  \begin{center}
    \includegraphics[width=0.90\textwidth]{src/assets/metodologia/apigateway_cliente}
    \label{fig:api_gateway_cliente}
  \end{center}
  \textit{Nota:} Adaptado de \textit{Microservices patterns: with examples in Java}, de \cite{richardson2018microservices}
  (p. 41). CC-BY-SA-4.0.
\end{figure}

\subsection{Retrospectiva}

Debido a que el proceso de diseño es un proceso iterativo, para poder identificar problemas
y posibles puntos de mejora para la siguiente iteración.
El fin de la retrospectiva es la mejora continua del proceso que a su vez mejora los productos
de seguir el proceso.

Este paso final es parecido a una retrospectiva Scrum y debido a esto es algo familiar para
el equipo de desarrollo por lo cual su implantación no debería conllevar problemas mayores.

Al finalizar este paso, se inicia nuevamente el proceso haciendo que la arquitectura
mejore con cada iteración.

\subsubsection*{Entradas}
No aplica, se hace un repaso general de el proceso acabado

\subsubsection*{Técnicas y herramientas}
\begin{enumerate}[a.]
  \item Reunión de retrospectiva.
  \item Lluvia de ideas.
\end{enumerate}

\subsubsection*{Salidas}
\begin{enumerate}
  \item Debilidades identificadas.
  \item Puntos a mejorar.
\end{enumerate}
