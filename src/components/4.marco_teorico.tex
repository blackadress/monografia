\mysection{4}{MARCO TEÓRICO}

\subsection{Antecedentes}

\subsection{Arquitectura Monolítica}

Según \cite{dmitry2014micro} una aplicación monolítica puede ser

\subsubsection{Ventajas de una Arquitectura de Microservicios}
\subsubsection{Desventajas de una Arquitectura de Microservicios}

\subsection{Arquitectura de Microservicios}

De acuerdo con \cite{dmitry2014micro}, un microservicio es un servicio ligero e independiente que
realiza funciones únicas y colabora con otros servicios similares utilizando una interface bien definida.
Una arquitectura basada en microservicios es un método para desarrollar una aplicación como un conjunto
de servicios pequeños e independientes. Cada uno de los servicios está siendo ejecutado bajo un
proceso independiente propio. Dichos servicios pueden comunicarse mediante mecanismos ligeros (usualmente
bajo HTTP). Servicios como estos pueden ser deplegados completamente independientes los unos de los otros.
Los servicios pueden estar escritos en diferentes lenguajes de programación, diferentes paradigmas,
usar sus propios modelos de datos, etcétera.

\cite{alshuqayran2016systematic} definen la arquitectura de microservicios como el estilo de arquitectura
que pone énfasis en dividir el sistema en servicios pequeños y ligeros que están construidos para
llevar a cabo una función de negocio de manera muy cohesiva.

\cite{newman2019monolith} indica que para definir una arquitectura de microservicios primero tenemos que
definir a los microservicios. Los microservicios son un conjunto de servicios independientemente
desplegables modelados alrededor de un dominio de negocio. Se comunican entre ellos mediante redes
y como una elección de arquitectura ofrecen varias opciones para resolver los problemas que un
equipo u organización pueden enfrentar. Por lo anterior indicado, se concluye que una arquitectura
de microservicios es una arquitectura basada en multiples microservicios trabajando en colaboración.

En resumen una arquitectura de microservicios se usa para construir sistemas distribuidos que

\subsubsection{Ventajas de una Arquitectura de Microservicios}
\subsubsection{Desventajas de una Arquitectura de Microservicios}

\subsection{Dominio de Negocio}
\subsection{Despliegue}

\subsection{Escalabilidad}

\subsection{Complejidad}

% \subsection{Integración continua}
