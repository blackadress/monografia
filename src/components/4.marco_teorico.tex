\mysection{4}{MARCO TEÓRICO}

\subsection{Antecedentes}

% \subsection{Arquitectura Monolítica}
%
% Según \cite{dmitry2014micro} una aplicación monolítica puede ser
%
% \subsubsection{Ventajas de una Arquitectura Monolítica}
%
% \subsubsection{Desventajas de una Arquitectura de Monolítica}


\subsection{Protocolo de Transferencia de Hipertexto (HTTP)}

Según \cite{goralski2017illustrated} el protocolo de transferencia de hipertexto (HTTP) es un protocolo de nivel de aplicación para sistemas de información
distribuidos, colaborativos y de hiper-media.
El protocolo es genérico, sin estado ({\it stateless}) que puede ser usado para muchas tareas más allá el uso para hipertexto.
Su extensibilidad se da por sus métodos de solicitud ({\it request methods}), códigos de error y cabeceras.

De acuerdo con \cite{gourley2002http} el HTTP es la lengua franca del Internet moderno.
HTTP es utilizado para enviar y recibir todo tipo de información de manera rápida, conveniente y confiable
desde servidores web de todo el mundo hacia los navegadores en las terminales de las personas.
El protocolo que usa HTTP para transmitir datos garantiza que la data no va a dañarse o perderse en
tránsito.


\subsection{Arquitectura de Microservicios}

De acuerdo con \cite{dmitry2014micro}, un microservicio es un servicio ligero e independiente que
realiza funciones únicas y colabora con otros servicios similares utilizando una interfaz bien definida.
Una arquitectura basada en microservicios es un método para desarrollar una aplicación como un conjunto
de servicios pequeños e independientes. Cada uno de los servicios está siendo ejecutado bajo un
proceso independiente propio. Dichos servicios pueden comunicarse mediante mecanismos ligeros (usualmente
bajo HTTP). Servicios como estos pueden ser desplegados completamente independientes los unos de los otros.
Los servicios pueden estar escritos en diferentes lenguajes de programación, diferentes paradigmas,
usar sus propios modelos de datos, etcétera.

Por su parte \cite{alshuqayran2016systematic} definen la arquitectura de microservicios como el estilo de arquitectura
que pone énfasis en dividir el sistema en servicios pequeños y ligeros que están construidos para
llevar a cabo una función de negocio de manera muy cohesiva.

Según el principio de responsabilidad única definido por \cite{martin2013clean} nos indica "juntar las
cosas que cambian por la misma razón, y separar las cosas que cambian por razones diferentes".
Los microservicios usan el mismo enfoque a los servicios independientes, trazando las fronteras
de cada servicio para coincidir con las fronteras del dominio del negocio para que así
se tengan unidades independientes y desacopladas.

Mientras que \cite{newman2019monolith} indica que para definir una arquitectura de microservicios primero tenemos que
definir a los microservicios. Los microservicios son un conjunto de servicios independientemente
desplegables modelados alrededor de un dominio de negocio. Se comunican entre ellos mediante redes
y como una elección de arquitectura ofrecen varias opciones para resolver los problemas que un
equipo u organización pueden enfrentar. Por lo anterior indicado, se concluye que una arquitectura
de microservicios es una arquitectura basada en múltiples microservicios trabajando en colaboración.

En resumen una arquitectura de microservicios está constituida por múltiples unidades que pueden ser
desplegadas independientemente las unas de las otras. Este conjunto de servicios han sido desarrollados
para solucionar problemas de un dominio de negocio.


\subsection{Servicio Web}

Para \cite{bruijn2006web} los servicios web son piezas de funcionalidad que son accesibles mediante la web.
La comunicación con estos se da mediante protocolos abiertos que se definen mediante estándares internacionales.
Diferentes clientes y aplicaciones desarrolladas en lenguajes de programación diferentes y ejecutándose
en sistemas operativos diferentes pueden comunicarse con los servicios web que implementan
los protocolos de comunicación.


\subsection{Microservicios}

Para su más fácil definición, hay que describir sus principales características.


\subsubsection{Pequeños y Enfocados en Hacer una Cosa y Hacerla Bien}

Una de las características principales de los microservicios es que son {\it pequeños}, este es un
punto de contención, ya que la definición de {\it pequeño} es problemática.
Según \cite{newman2021building}, no podemos asignar la etiqueta de {\it pequeño} basándonos en líneas
de código ya que diferentes lenguajes de programación son más o menos expresivos.
Además de que los dominios de problema pueden ser tener una complejidad elevada lo cual hace imposible
expresar su solución en pocas líneas de código.

Debido a lo anterior mencionado, Newman indica que son los dueños del código (los desarrolladores)
los encargados de definir {pequeño} según el contexto.
Cabe mencionar que los programadores entienden intuitivamente cuando el sistema en el que están trabajando
es {\it muy grande}.

Por su lado \cite{kleppmann2015kafka} menciona el cómo la filosofía UNIX influyó al diseño de la arquitectura
de microservicios:
\begin{itemize}
  \item Hacer que cada programa haga una cosa y que la haga bien.
  \item Hacer que los programas puedan trabajar juntos
  \item Hacer que los programas puedan mantener inputs de textos, porque es una interfaz universal
\end{itemize}

Con respecto al tercer punto, en lugar de hacer las comunicaciones mediante {\it standard input},
los microservicios se comunican mediante HTTP.


\subsubsection{Autónomos}

De acuerdo con \cite{newman2019monolith}, los microservicios son entidades separadas.
Cada microservicio puede ser desplegado como un servicio aislado o puede ser su propio proceso
de un sistema operativo.
Estos servicios tienen que tener la capacidad de ser cambiados independientemente de sus pares
y ser desplegados de manera tal que los clientes que consumen el servicio no necesiten de cambiar.

Mientras que \cite{nadareishvili2016microservice} indican que la autonomía de servicios es posible
gracias al diseño de interoperabilidad mediante comunicación vía mensajería.
Debido a lo anteriormente mencionado, los microservicios son necesariamente una forma de una
arquitectura basada en servicios (SOA por sus siglas en inglés); además de que, debido a la
complejidad en el despliegue de microservicios es casi una necesidad el implementar prácticas de
DevOps (desarrollo y operaciones) para la automatización del despliegue.


\subsection{Acoplamiento}

Según \cite{newman2019monolith} el acoplamiento es la característica de un programa que mide el cómo
cambiar código en un lugar requiere cambiar el programa en otro lugar.
Los dos tipos de acoplamiento más comunes son: el acoplamiento de implementación y el acoplamiento temporal.
El acoplamiento de implementación suele darse cuando componentes comparten un flujo de datos y es el más
sencillo de solucionar.
El acoplamiento temporal tiene que ver con el momento de ejecución y es uno de los principales
desafíos de hacer llamadas síncronas en un entorno distribuido.

De acuerdo con \cite{percival2020architecture} el acoplamiento es cuando los desarrolladores no
pueden cambiar el componente A por miedo a romper el componente B. También indican que el
acoplamiento a nivel local es algo bueno porque es un signo que el módulo ha sido agrupado correctamente.
De igual manera advierten de los problemas que trae el tener acoplamiento a nivel global que trae
consigo el incremento de los riesgos y costos de cambiar el código base.


\subsection{Cohesión}

De acuerdo con \cite{newman2019monolith} la cohesión describe el cómo se agrupa el código relacionado,
básicamente: {\it el código que cambia en conjunto, se agrupa}.
El organizar el código base utilizando la cohesión hace que las modificaciones que inevitablemente
va a necesitar, sea una operación sencilla.
Cuando la cohesión se respeta el equipo de desarrollo puede moverse rápidamente debido a que no tiene
que buscar lógica en múltiples lugares, sino que solo en un único módulo.

Mientras que \cite{percival2020architecture}

\subsection{Ley de Conway y La Importancia de la Comunicación}

Existen patrones en las arquitecturas utilizadas por cada organización, estas observaciones
las hizo \cite{conway1968committees}: ``Cualquier organización que diseñe un sistema producirá
un diseño que copia la estructura de comunicación de dicha organización''

% \begin{quote}
% ``Cualquier organización que diseñe un sistema producirá un diseño que copia la estructura de
% comunicación de dicha organización''
% \end{quote}

De acuerdo a estas observaciones \cite{newman2019monolith} asevera que la arquitectura de tres capas
(MVC) surge de la forma en la cual organizaciones de TI agrupaban a las personas: se creaban grupos
separados de administradores de bases de datos, desarrolladores de infraestructura y desarrolladores
de interfaces gráficas. Entonces concluye que si agrupamos a las personas con respecto a sus competencias
principales, estas personas organizarán el software de manera similar.

Actualmente la forma más común de agrupar personas en organizaciones de TI es en equipos multidisciplinarios
armando equipos ágiles que tienen la responsabilidad de lanzar software de calidad lo más rápido posible.
Siguiendo esta evolución natural de los equipos de desarrollo, Newman concluye que los microservicios,
además de ser una arquitectura de software, reflejan la forma de comunicación de la organización
que los implementa.

De igual forma, \cite{nadareishvili2016microservice} indican que, en términos sencillos, la
{\it la comunicación dicta el resultado}, por lo cual hacen hincapié en que la calidad de un sistema
depende casi exclusivamente de los canales de comunicación que establece la organización que lo implementa.
Por lo tanto, las organizaciones que quieren aplicar la arquitectura de microservicios, necesitan
organizar los equipos que trabajarán en el producto de tal manera que su comunicación sea similar a 
la arquitectura.


\subsection{Despliegue}


\subsubsection{Despliegue Acoplado}

\cite{newman2019monolith} menciona que el despliegue acoplado se encuentra en sistemas en los que el
despliegue de un servicio, necesita del despliegue de otro servicio.
Entonces, se entiende que el hacer cambios, por más minúsculos que sean, involucran a todos los módulos
de la aplicación.

Según \cite{nadareishvili2016microservice} el despliegue acoplado se da cuando existe la necesidad
de compartir data entre módulos.
También indica que una metodología de cascada suele llevar a que la implementación del sistema
quede fuertemente acoplado.


\subsubsection{Despliegue Independiente}

De acuerdo con \cite{newman2019monolith} el despliegue independiente es la idea de que podemos
hacer un cambio a un microservicio y desplegarlo a un ambiente de producción sin tener que utilizar
ningún otro servicio.
En este tipo de despliegue, un cambio puede ser desplegado sin necesidad de desplegar todo el sistema,
sino solo el módulo o servicio que sufrió el cambio.

Según \cite{nadareishvili2016microservice} el despliegue independiente es un principio fundamental
en el estilo arquitectónico de microservicios.
En pocas palabras, el despliegue independiente es la posibilidad de poder desplegar un 
servicio de manera tal que no dependa de ningún otro servicio.
Esto permite que sea posible el llevar a cabo escalabilidad selectiva o bajo demanda.


\subsection{Dominio de Negocio}

De acuerdo con \cite{bosselmann2014domain} para definir un dominio de negocio de manera formal,
no existe una ontología formal adecuada para los desarrolladores de software.
Debido a esto solo se puede definir el dominio de negocio como lo siguiente: una descripción
del valor que una compañía ofrece a uno o varios segmentos de clientes y la arquitectura de
la firma y su red de socios para crear, publicitar y entregar su valor y capital relacional,
con el fin de generar flujos de ingresos rentables y sostenibles.

Por su parte \cite{heidari2021ontology} definen al dominio de negocio como las esferas específicas
para las soluciones orientadas a negocios.
El concepto se refiere a los aspectos en el mundo real de la solución ofrecida por el negocio o empresa.
Para los desarrolladores, el dominio de negocio indica los requerimientos y los criterios de aceptación
para el sistema a desarrollar.
Estos criterios pueden indicar un nivel alto de segregación para diferentes áreas.

Según \cite{khononov2021learning} un dominio de negocio define el principal área de actividad de
la empresa. En general es el servicio que la empresa provee a sus clientes.
Una empresa puede operar en múltiples dominios de negocio.


\subsection{Escalabilidad}


\subsection{Complejidad}


\subsection{Sistema Distribuido}

% \subsection{Integración continua}
